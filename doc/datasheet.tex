\documentclass[10pt,a4paper,onecolumn,notitlepage]{article}
\usepackage[utf8]{inputenc}
\usepackage{amsmath}
\usepackage{amsfonts}
\usepackage{amssymb}
\usepackage{hyperref}
\author{Stefan Kraatz}
\begin{document}
\section*{presenter and tool}
\begin{description}
\item[Student:] Stefan Kraatz 
\item[email:] stefan.kraatz@b-tu.de
\item[Tool name (long/short):] CPPUnit/ CppUnit - C++ port of JUnit
\end{description}
\section*{tool overview}
\begin{description}
\item[Source:]
\item[Target Language(s):] C++
\item[Platforms(s):] all major operating systems
\item[License Status:] LGPL
\end{description}
\subsection*{functionality in a nutshell}
CppUnit is a framework, that provides library support for creating test cases for C++ code (whitebox testing and blackbox testing of libraries). It help the developer in running test suites composed of test cases and assists in collecting the results and presenting them in parseable output (XML). It does not assist in creating the test data for the unit tests. The resulting code can be compiled, executed and the results evaluated manually or inside tools.
\section*{Experiences}
\begin{description}
\item[Installation] compile or use system package manager
\item[Stability] very stable (used for big projects as the unit test framework of choice)
\item[Usability] good, but probably depends on requirements (xml output needs converting for some tools)
\item[UserInterface]: Ascii but provides classes for integrating it into graphical environments
\item[Learning Curve]: for someone with some exposure to unit testing such as JUnit, easy, not easy for absolute beginners 
\item[Installation] : compile or install via package system
\item[Else:] cppunit is relatively old and newer frameworks are preferred by tool developers, such that support may be lacking for the tool of choice. 
\end{description}
\section*{References}
\subsection*{slides:}
\begin{itemize}
\item \hyperref[https://www.slideshare.net/mudabbirwarsi/cpp-unit-28dec]{https://www.slideshare.net/mudabbirwarsi/cpp-unit-28dec}
\item 
\end{itemize}
\subsection*{webpages:}

\end{document}